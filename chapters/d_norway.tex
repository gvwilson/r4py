\chapter{Why We Can't Have Nice Things}\label{norway}

In the beginning, there were tables:
rows upon rows,
with columns separated by commas or tabs or something more exotic.
They were elegant but limited,
so programmers said,
``Let there be XML!''
And lo,
there came a great wailing and gnashing of teeth,
for who among us can truly comprehend external (parsed) general entity declarations
and the encoding thereof?

But there came to programmers a new hope called \href{https://en.wikipedia.org/wiki/JSON}{JSON},
and JSON said,
``Look, folks, why don't you just write data the way you write data structures in programs?''
And programmers looked upon JSON,
with its lists of numbers and strings and name/value pairs,
and said,
``Cool!  But what about comments?  And do I really have to quote all those strings?''
And so programmers turned to \href{https://en.wikipedia.org/wiki/YAML}{YAML},
which permits the commenting of data,
and the unquoting of strings,
and much else that is good and welcome to those who believe that
if you couldn't have done it in a line-editor in 1970,
you shouldn't be doing it today.

But a darkness lurked within the heart of YAML,
for in striving to make life easy for its users,
it had left a slippery puddle of grease before the very gates of Hell.
In place of JSON's parentheses and quotes:

\begin{lstlisting}
{"instructors": ["Canada", "Denmark", "France"],
 "workshops": ["Brazil", "Norway", "Poland"]}
\end{lstlisting}

\noindent
YAML, ever gracious, allowed programmers to write:

\begin{lstlisting}
instructors:
- Canada
- Denmark
- France
workshops:
- Brazil
- Norway
- Poland
\end{lstlisting}

\noindent
And programmers looked upon it and said it was good,
for who truly likes typing the brace that is curly?

But then one programmer said,
``Cool!
Although we really should use ISO-3166 country codes instead of countries' names,
because standards and internationalization.''
And upon saying that,
he re-wrote the data as:

\begin{lstlisting}
instructors:
- ca
- de
- fr
workshops:
- br
- no
- pl
\end{lstlisting}

But upon this change there came great puzzlement,
because this configuration file was used to select flag icons to display upon a map,
and among those flags were many countries, but not Norway.
And after much scratching of heads and googling of terms,
it was discovered that YAML interprets the unquoted word \texttt{no} as meaning \texttt{false}.
Upon learning which,
the programmer blasphemed mightily and at great length,
repeating also,
``Seriously?  Seriously??''

For inasmuch as it may be convenient for some to allow plain English synonyms for truth and falsity,
each such accommodation is a trap from which the unwary will emerge wearied and cynical,
if they emerge at all.
And all programmers know this,
having had their own spirits widdled upon by the inconsistencies of others
as a senile old poodle does widdle upon its owner's leg,
mistaking it (or not) for a fire hydrant.
All programmers know this,
but persist in committing these same sins themselves.

And this is why we cannot have nice things.
