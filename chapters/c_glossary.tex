\chapter{Glossary}\label{glossary}

\begin{description}

\gitem{g:absolute-row-number}{Absolute row number}
The sequential index of a row in a table,
regardless of what sections of the table is being displayed.

\gitem{g:aggregation}{Aggregation}
To combine many values into one,
e.g.,
by summing a set of numbers or concatenating a set of strings.

\gitem{g:alias}{Alias}
To have two (or more) references to the same physical data.

\gitem{g:anonymous-function}{Anonymous function}
A function that has not been assigned a name.
Anonymous functions are usually quite short,
and are usually defined where they are used,
e.g.,
as callbacks.

\gitem{g:attribute}{Attribute}
A name-value pair associated with an object,
used to store metadata about the object
such as an array's dimensions.

\gitem{g:catch-exception}{Catch (exception)}
To accept responsibility for handling an error
or other unexpected event.
R prefers ``\grefcross{g:handle-condition}{handling} a \grefcross{g:condition}{condition}''
to ``catching an \grefcross{g:exception}{exception}''.

\gitem{g:condition}{Condition}
An error or other unexpected event that disrupts the normal flow of control.
See also \grefcross{g:handle-condition}{handle}.

\gitem{g:constructor}{Constructor (class)}
A function that creates an object of a particular class.
In the \grefcross{g:S3}{S3} object system,
constructors are a convention rather than a requirement.

\gitem{g:copy-on-modify}{Copy-on-modify}
The practice of creating a new copy of \grefcross{g:alias}{aliased} data
whenever there is an attempt to modify it
so that each reference will believe theirs is the only one.

\gitem{g:data-frame}{Data frame}
A two-dimensional data structure
in which columns represent values that were observed
and rows represent observations.
Each column has a unique name;
all of the values in a particular column must be of the same type,
but the columns themselves may be of different types.

\gitem{g:double-square-brackets}{Double square brackets}
An index enclosed in \texttt{[[...]]},
used to return a single value of the underlying type.
See also \grefcross{g:single-square-brackets}{single square brackets}.

\gitem{g:eager-evaluation}{Eager evaluation}
Evaluating an expression as soon as it is formed.

\gitem{g:empty-vector}{Empty vector}
A vector that contains no elements.
Empty vectors have a type such as logical or character,
and are \emph{not} the same as \grefcross{g:null}{null}.

\gitem{g:environment}{Environment}
A structure that stores a set of variable names and the values they refer to.

\gitem{g:error}{Error}
The most severe type of built-in \grefcross{g:condition}{condition} in R.

\gitem{g:evaluating-function}{Evaluating function}
A function that takes arguments as values.
Most functions are evaluating functions.

\gitem{g:evaluation}{Evaluation}
The process of taking a complex expression such as \texttt{1+2*3/4}
and turning it into a single irreducible value.

\gitem{g:exception}{Exception}
An object containing information about an error,
or the condition that led to the error.
R prefers ``\grefcross{g:handle-condition}{handling} a \grefcross{g:condition}{condition}''
to ``\grefcross{g:catch-exception}{catching} an \grefcross{g:exception}{exception}''.

\gitem{g:export}{Export (from a package)}
To make something visible to programs using that package.
Packages often contain helper functions or lookup tables that are no one's business but their own;
requiring package authors to export things explicitly
provides a modicum of privacy.

\gitem{g:filter}{Filter}
To choose a set of records according to the values they contain.

\gitem{g:fully-qualified-name}{Fully qualified name}
An unambiguous name of the form package::thing.

\gitem{g:functional-programming}{Functional programming}
A style of programming in which functions transform data rather than modifying it.
Functional programming relies heavily on \grefcross{g:higher-order-function}{higher-order functions}.

\gitem{g:generic-function}{Generic function}
A collection of functions with similar purpose,
each operating on a different class of data.

\gitem{g:global-environment}{Global environment}
The \grefcross{g:environment}{environment} that holds top-level definitions in R,
e.g.,
those written directly in the interpreter.

\gitem{g:group}{Group}
To divide data into subsets according to some criteria
while leaving records in a single structure.

\gitem{g:handle-condition}{Handle (a condition)}
To accept responsibility for handling an error
or other unexpected event.
R prefers ``handling a \grefcross{g:condition}{condition}''
to ``\grefcross{g:catch-exception}{catching} an \grefcross{g:exception}{exception}''.

\gitem{g:helper}{Helper (class)}
In \grefcross{g:S3}{S3},
a function that \grefcross{g:constructor}{constructs} and \grefcross{g:validator}{validates}
an instance of a class.

\gitem{g:heterogeneous}{Heterogeneous}
Potentially containing data of different types.
Most vectors in R are \grefcross{g:homogeneous}{homogeneous},
but lists can be heterogeneous.

\gitem{g:higher-order-function}{Higher-order function}
A function that takes one or more other functions as parameters.
Higher-order functions such as \texttt{map} are commonly used in \grefcross{g:functional-programming}{functional programming}.

\gitem{g:homogeneous}{Homogeneous}
Containing data of only a single type.
Most vectors in R are homogeneous.

\gitem{g:hubris}{Hubris}
Excessive pride or self-confidence.
See also \grefcross{g:unit-test}{unit test} (lack of).

\gitem{g:iso3-country-code}{ISO3 country code}
A three-letter code defined by ISO 3166-1 that identifies a specific country,
dependent territory,
or other geopolitical entity.

\gitem{g:lazy-evaluation}{Lazy evaluation}
Delaying evaluation of an expression until the value is actually needed
(or at least until after the point where it is first encountered).

\gitem{g:list}{List}
A vector that can contain values of many different types.

\gitem{g:list-comprehension}{List comprehension}
An expression that generates a new list from an existing one via an implicit loop.

\gitem{g:logical-indexing}{Logical indexing}
To index a vector or other structure with a vector of Booleans,
keeping only the values that correspond to true values.

\gitem{g:message}{Message}
The least severe type of built-in \grefcross{g:condition}{condition} in R.

\gitem{g:method}{Method}
An implementation of a \grefcross{g:generic-function}{generic function}
that handles objects of a specific class.

\gitem{g:NA}{NA}
A special value used to represent data that is Not Available.

\gitem{g:name-collision}{Name collision}
A situation in which the same name has been used in two different packages
which are then used together,
leading to ambiguity.

\gitem{g:named-list}{Named list}
A list whose elements have been given names.
After assigning \texttt{list(red=128, green=0, blue=255)} to the variable \texttt{color},
we can get the value 128 using either \texttt{color\$red} or \texttt{color[[1]]}.

\gitem{g:negative-selection}{Negative selection}
To specify the elements of a vector or other data structure that \emph{aren't} desired
by negating their indices.

\gitem{g:null}{Null}
A special value used to represent a missing object.
\texttt{NULL} is not the same as \texttt{NA},
and neither is the same as an \grefcross{g:empty-vector}{empty vector}.

\gitem{g:package}{Package}
A collection of code, data, and documentation
that can be distributed and re-used.

\gitem{g:pipe-operator}{Pipe operator}
The \texttt{\%\textgreater{}\%} used to make the output of one function the input of the next.

\gitem{g:prefix-operator}{Prefix operator}
An operator that comes before the single value it operates on,
such as the \texttt{-} in \texttt{-(a*b)}.

\gitem{g:promise}{Promise}
A data structure used to record an unevaluated expression for lazy evaluation.

\gitem{g:pull-indexing}{Pull indexing}
Vectorized indexing in which the value at location \emph{i} in the index vector
specifies which element of the source vector
is being pulled into that location in the result vector,
i.e., \texttt{result[i]\ =\ source[index[i]]}.
See also \grefcross{g:push-indexing}{push indexing}.

\gitem{g:push-indexing}{Push indexing}
Vectorized indexing in which the value at location \emph{i} in the index vector
specifies an element of the result vector that gets the corresponding element of the source vector,
i.e., \texttt{result[index[i]]\ =\ source[i]}.
Push indexing can easily produce gaps and collisions.
See also \grefcross{g:pull-indexing}{pull indexing}.

\gitem{g:quosure}{Quosure}
A data structure containing an unevaluated expression and its environment.

\gitem{g:quoting-function}{Quoting function}
A function that is passed expressions rather than the values of those expressions.

\gitem{g:raise-exception}{Raise (exception)}
A way of indicating that something has gone wrong in a program,
or that some other unexpected event has occurred.
R prefers ``\grefcross{g:signal-condition}{signalling} a \grefcross{g:condition}{condition}''
to ``raising an \grefcross{g:exception}{exception}''.

\gitem{g:range-expression}{Range expression}
An expression of the form low:high
that is transformed a sequence of consecutive integers.

\gitem{g:reactive-programming}{Reactive programming}
A style of programming in which actions are triggered by external events.

\gitem{g:reactive-variable}{Reactive variable}
A variable whose value is automatically updated when some other value or values change.

\gitem{g:recycle}{Recycle}
To re-use values from a shorter vector in order to generate
a sequence of the same length as a longer one.

\gitem{g:regular-expression}{Regular expression}
A pattern for matching text.
Regular expressions are themselves written as text,
which makes them as cryptic as they are powerful.

\gitem{g:relative-row-number}{Relative row number}
The index of a row in a displayed portion of a table,
which may or may not be the same as the \grefcross{g:absolute-row-number}{absolut row number}
within the table.

\gitem{g:repository}{Repository}
The place where a version control system stores a project's files
and the metadata used to record their history.

\gitem{g:S3}{S3}
A framework for object-oriented programming in R.

\gitem{g:scalar}{Scalar}
A single value of a particular type, such as 1 or ``a''.
Scalars don't really exist in R;
values that appear to be scalars are actually vectors of unit length.

\gitem{g:select}{Select}
To choose entire columns from a table by name or location.

\gitem{g:testing-setup}{Setup (testing)}
Code that is automatically run once before each \grefcross{g:unit-test}{unit test}.

\gitem{g:signal-condition}{Signal (a condition)}
A way of indicating that something has gone wrong in a program,
or that some other unexpected event has occurred.
R prefers ``signalling a \grefcross{g:condition}{condition}''
to ``\grefcross{g:raise-exception}{raising} an \grefcross{g:exception}{exception}''.

\gitem{g:single-square-brackets}{Single square brackets}
An index enclosed in \texttt{[...]},
used to select a structure from another structure.
See also \grefcross{g:double-square-brackets}{double square brackets}.

\gitem{g:storage-allocation}{Storage allocation}
Setting aside a block of memory for future use.

\gitem{g:testing-teardown}{Teardown (testing)}
Code that is automatically run once after each \grefcross{g:unit-test}{unit test}.

\gitem{g:test-fixture}{Test fixture}
The data structures, files, or other artefacts on which a \grefcross{g:unit-test}{unit test} operates.

\gitem{g:test-runner}{Test runner}
A software tool that finds and runs \grefcross{g:unit-test}{unit tests}.

\gitem{g:tibble}{Tibble}
A modern replacement for R's data frame,
which stores tabular data in columns and rows,
defined and used in the \grefcross{g:tidyverse}{tidyverse}.

\gitem{g:tidyverse}{Tidyverse}
A collection of R packages for operating on tabular data in consistent ways.

\gitem{g:unit-test}{Unit test}
A function that tests one aspect or property of a piece of software.

\gitem{g:validator}{Validator (class)}
A function that checks the consistency of an \grefcross{g:S3}{S3} object.

\gitem{g:variable-arguments}{Variable arguments}
In a function,
the ability to take any number of arguments.
R uses \texttt{...} to capture the ``extra'' arguments.

\gitem{g:vector}{Vector}
A sequence of values,
usually of \grefcross{g:homogeneous}{homogeneous} type.
Vectors are \emph{the} fundamental data structure in R;
\grefcross{g:scalar}{scalars} are actually vectors of unit length.

\gitem{g:vectorize}{Vectorize}
To write code so that operations are performed on entire vectors,
rather than element-by-element within loops.

\gitem{g:warning}{Warning}
A built-in \grefcross{g:condition}{condition} in R of middling severity.

\gitem{g:widget}{Widget}
An interactive control element in an user interface.

\end{description}
