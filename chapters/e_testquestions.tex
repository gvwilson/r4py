\chapter{Comparing Tests}\label{testquestions}

Please see page~\ref{rectangle-overlap} for the background to this discussion.

The function \texttt{rectangleOverlap} takes two rectangles as input
and returns a rectangle representing their overlap.
Each rectangle is represented by a vector $(x_{low}, y_{low}, x_{high}, y_{high})$.
Some interesting test cases include:

\begin{tabular}{l|l|l}
  \textbf{Left}
  & \textbf{Right}
  & \textbf{Output}
  \\

  \texttt{0.0, 0.0, 2.0, 2.0)}
  & \texttt{"a string"}
  & Error.
  \\
  \multicolumn{3}{l}{Function only works for rectangles.}
  \\

  \texttt{0.0, 0.0, 2.0)}
  & \texttt{1.0, 1.0, 3.0, 3.0)}
  & Error.
  \\
  \multicolumn{3}{l}{Function only works for \emph{complete} rectangles.}
  \\

  \texttt{0.0, 0.0, 2.0, 2.0)}
  & \texttt{1.0, 1.0, 3.0, 3.0)}
  & \texttt{1.0, 1.0, 2.0, 2.0)}
  \\
  \multicolumn{3}{l}{Partial overlap.}
  \\

  \texttt{1.0, 1.0, 2.0, 2.0)}
  & \texttt{1.0, 1.0, 2.0, 2.0)}
  & \texttt{1.0, 1.0, 2.0, 2.0)}
  \\
  \multicolumn{3}{l}{Exact overlap.}
  \\

  \texttt{0.0, 0.0, 10.0, 10.0)}
  & \texttt{1.0, 1.0, 3.0, 3.0)}
  & \texttt{1.0, 1.0, 3.0, 3.0)}
  \\
  \multicolumn{3}{l}{Wholly contained.}
  \\

  \texttt{0.0, 0.0, 1.0, 1.0)}
  & \texttt{1.0, 0.0, 2.0, 1.0)}
  & \texttt{1.0, 0.0, 1.0, 1.0)}
  \\
  \multicolumn{3}{l}{Overlap on an edge (a zero-width rectangle).}
  \\

  \texttt{0.0, 0.0, 1.0, 1.0)}
  & \texttt{1.0, 1.0, 2.0, 2.0)}
  & \texttt{1.0, 1.0, 1.0, 1.0)}
  \\
  \multicolumn{3}{l}{Overlap at a point (a zero-area rectangle).}
  \\

  \texttt{0.0, 0.0, 1.0, 1.0)}
  & \texttt{2.0, 2.0, 3.0, 3.0)}
  & ??
  \\
  \multicolumn{3}{l}{Well this is interesting{\ldots}}
  \\
\end{tabular}

The first two cases seem reasonable:
the function can't be expected to work for invalid input.
The next three are also reasonable:
rectangles can overlap in a variety of ways,
and we should check one instance of each.

But suddenly we find ourselves paddling in deceptively calm waters
while something tentacular stirs beneath us.
If two rectangles align on an edge, does that count as an overlap?
Roughly half of people would say ``yes,''
and would then (perhaps reluctantly) agree that by implication,
two rectangles whose corners touch ``overlap'' at a point with zero with and zero height.
However,
if we ask the questions in the reverse order,
most people say that touching at a corner doesn't count,
and then argue that touching on an edge must not count either.

But if edges and corners don't count---if we insist that two rectangles overlap
if and only if the intersection of their areas is non-empty,
which sounds so mathematical that it must be the right---then
what should \texttt{rectangleOverlap} return in those cases?
What should it return when its inputs obviously and indisputably don't overlap at all?
This isn't an error,
any more than asking whether the word ``gibbous'' contains the letter 'a' is an error.
Should we use \texttt{NULL} to represent an empty rectangle?
Should we use a special value such as \texttt{c(0, 0, 0, 0)}?
And if so,
why those four coordinates,
and is this consistent with what the rest of our rectangle manipulators do?
Luckily for us,
these questions are independent of the programming language we are using,
and therefore outside the scope of this book.
